%!TEX TS-program = xelatex

% Шаблон документа LaTeX создан в 2018 году
% Алексеем Подчезерцевым
% В качестве исходных использованы шаблоны
% 	Данилом Фёдоровых (danil@fedorovykh.ru) 
%		https://www.writelatex.com/coursera/latex/5.2.2
%	LaTeX-шаблон для русской кандидатской диссертации и её автореферата.
%		https://github.com/AndreyAkinshin/Russian-Phd-LaTeX-Dissertation-Template

\documentclass[a4paper,14pt]{article}

\input{data/preambular.tex}
\begin{document} % конец преамбулы, начало документа
    \begin{titlepage}
    \begin{center}
        ФЕДЕРАЛЬНОЕ ГОСУДАРСТВЕННОЕ АВТОНОМНОЕ \\
        ОБРАЗОВАТЕЛЬНОЕ УЧРЕЖДЕНИЕ ВЫСШЕГО ОБРАЗОВАНИЯ\\
        «НАЦИОНАЛЬНЫЙ ИССЛЕДОВАТЕЛЬСКИЙ УНИВЕРСИТЕТ\\
        «ВЫСШАЯ ШКОЛА ЭКОНОМИКИ»
    \end{center}

    \begin{center}
        \textbf{Московский институт электроники и математики}

        \textbf{им. А.Н.Тихонова НИУ ВШЭ}

        \vspace{2ex}

        \textbf{Департамент компьютерной инженерии}
    \end{center}
    \vspace{1ex}

    \begin{center}
        Курс «Системы искусственного интеллекта»
    \end{center}


    \begin{center}
        \textbf{ОТЧЕТ\\
        ПО ЗАДАЧАМ НА ПРОЛОГЕ
        }
    \end{center}


    \vspace{30ex}

    \begin{flushright}
        \textbf{Выполнил:}

        \vspace{2ex}

        Студент группы БИВ174

        \vspace{2ex}

        Подчезерцев Алексей Евгеньевич


    \end{flushright}

    \vfill
    \begin{center}
        Москва \the\year \, г.
    \end{center}

\end{titlepage}
\addtocounter{page}{1}
    \section*{Задание 1}

    На базе имеющихся предикатов, описывающих родственные отношения: man(symbol), woman(symbol), parent(symbol,symbol), Kol\_det(symbol, symbol, integer), mother(symbol), father(symbol).
    Определить предикаты брат, сестра, дядя, бабушка.

    Исходный код:
    {\small \VerbatimInput{code/task_01.pl}}
    Вывод программы:
    {\small \VerbatimInput{logs/log_01.log}}


    \section*{Задание 2}

    Имеется информационная база о служащих предприятия в виде:
    идентификационный номер служащего, фамилия, должность, зарплата, номер отдела. Дополнительно отдельно хранятся данные о семейном положении служащего в виде: идентификационный номер служащего, фамилия жены или мужа, число детей.
    Написать программу, содержащую пример такой базы и дополнить её правилами:
    \begin{enumerate}

        \item А знаком с В , если они работают в одном отделе
        \item Жена/муж А знаком с женой/мужем В , если А и В работают в одном отделе.
        \item Составить целевые запросы к программе:
        \item Выдать всех служащих первого отдела
        \item Определить работников, получающих зарплату более заданной
        \item Определить работников, имеющих более одного ребёнка.
    \end{enumerate}


    Исходный код:
    {\small \VerbatimInput{code/task_02.pl}}
    Вывод программы:
    {\small \VerbatimInput{logs/log_02.log}}


    \section*{Задание 3}

    Составить программу, вычисляющую значение функции
    $f(x,y)=3 \cdot y \cdot a(x) + sin(x \cdot y) \cdot b(y)$, где
    $a(x)=\sqrt{| x|}$,
    $b(y)=\cos(y) +2$.
    Вывести промежуточные значения $a(x)$,  $b(y)$   и результат      $f (x,y)$.
    Использовать два способа -- вычисление в одном предикате и выделение нахождения $a(x)$ и  $b(y)$, в отдельные предикаты.


    Исходный код:
    {\small \VerbatimInput{code/task_03.pl}}
    Вывод программы:
    {\small \VerbatimInput{logs/log_03.log}}


    \section*{Задание 4}

    Написать программу, вычисляющую значение функции вида:
    \[
        f(x,y)=
        \begin{cases}
            2 \cdot x& \text{если } x+y < -1\\
            \cos(x \cdot y) &\text{если } -1 \le x+y \le 1\\
            \sqrt{x + y} &\text{если } x+y > 1\\
        \end{cases}
    \]
    Исходный код:
    {\small \VerbatimInput{code/task_04.pl}}
    Вывод программы:
    {\small \VerbatimInput{logs/log_04.log}}


    \section*{Задание 5}

    Написать программу, осуществляющую сравнение двух чисел А и В и формирующую одно из сообщений:
    <<А больше В>>, <<А меньше В>>, <<А равно В>>.

    Исходный код:
    {\small \VerbatimInput{code/task_05.pl}}
    Вывод программы:
    {\small \VerbatimInput{logs/log_05.log}}


    \section*{Задание 6}

    Составить предикат  $max1(x,y,z)$, определяющий max из двух чисел x и y.


    Исходный код:
    {\small \VerbatimInput{code/task_06.pl}}
    Вывод программы:
    {\small \VerbatimInput{logs/log_06.log}}


    \section*{Задание 7}

    Составить программу для решения квадратного уравнения  $ax^2+bx+c=0 $

    Исходный код:
    {\small \VerbatimInput{code/task_07.pl}}
    Вывод программы:
    {\small \VerbatimInput{logs/log_07.log}}


    \section*{Задание 8}

    Имеются сведения о возрасте различных людей в форме:
    возраст(фамилия, число\_лет) и их поле: мужчина(фамилия), женщина(фамилия).
    Используя внутрипрограммную цель получить список всех мужчин старше 18 лет.

    Исходный код:
    {\small \VerbatimInput{code/task_08.pl}}
    Вывод программы:
    {\small \VerbatimInput{logs/log_08.log}}


    \section*{Задание 9}

    Имеются данные о том, кто кого победил во время соревнований в виде: победил(фамилия, фамилия).
    Используя предикат победил и предикат cut, необходимо классифицировать соревновавшихся следующим образом:
    если Х победил кого-либо и Х был кем-то побеждён, то Х -- боец
    иначе если Х победил кого-либо, то Х -- победитель
    иначе если Х был кем-то побеждён, то Х -- спортсмен.

    Исходный код:
    {\small \VerbatimInput{code/task_09.pl}}
    Вывод программы:
    {\small \VerbatimInput{logs/log_09.log}}


    \section*{Задание 10}

    Пусть имеется информация о возрасте некоторых людей в виде:
    возраст(имя, число лет) и их поле: мужчина(имя), женщина(имя), цвете волос: цвет\_волос(имя, цвет) а также о различных городах, когда-либо посещавшихся данными людьми. Необходимо составить программу, осуществляющую поиск и выбор одного рыжеволосого мужчины старше 18 лет из списка и вывод всех городов, которые он посещал. Целевой запрос должен быть внутрипрограммным.

    Исходный код:
    {\small \VerbatimInput{code/task_10.pl}}
    Вывод программы:
    {\small \VerbatimInput{logs/log_10.log}}


    \section*{Задание 11}

    Составить программу, определяющую х в степени n, где х, n - целые числа, $n \ge 0$.

    Исходный код:
    {\small \VerbatimInput{code/task_11.pl}}
    Вывод программы:
    {\small \VerbatimInput{logs/log_11.log}}


    \section*{Задание 12}

    Последовательность Фибоначчи имеет вид:    $0,1,1,2,3,5,8\dots$ и определяется формулами:
    $f(1) = 0, f(2) = 1, f(k )= f(k-2) + f(k-1)$
    Составить программу, определяющую число Фибоначчи с заданным номером.

    Исходный код:
    {\small \VerbatimInput{code/task_12.pl}}
    Вывод программы:
    {\small \VerbatimInput{logs/log_12.log}}


    \section*{Задание 13}

    Имеются факты, описывающие родственные отношения между людьми, указывая, кто является чьим родителем:
    родитель(X,Y), т.е. X -- родитель Y.
    Описать предикат, позволяющий определить, кто является чьим потомком:
    потомок(А,В), т.е. А~--~потомок В.


    Исходный код:
    {\small \VerbatimInput{code/task_13.pl}}
    Вывод программы:
    {\small \VerbatimInput{logs/log_13.log}}


    \section*{Задание 14}

    Определить предикат lensp(N,L) для нахождения длины списка L.

    Исходный код:
    {\small \VerbatimInput{code/task_14.pl}}
    Вывод программы:
    {\small \VerbatimInput{logs/log_14.log}}


    \section*{Задание 15}

    Составить программу для поиска суммы всех четных элементов списка.


    Исходный код:
    {\small \VerbatimInput{code/task_15.pl}}
    Вывод программы:
    {\small \VerbatimInput{logs/log_15.log}}


    \section*{Задание 16}

    Составить программу для перевода списка, записанного в виде цифр, в список из соответствующих слов, например:
    [3,5,1,3]  преобразовать в [“три”, “пять”, “один”, “три”].


    Исходный код:
    {\small \VerbatimInput{code/task_16.pl}}
    Вывод программы:
    {\small \VerbatimInput{logs/log_16.log}}


    \section*{Задание 17}

    Написать программу для определения максимального значения в списке.


    Исходный код:
    {\small \VerbatimInput{code/task_17.pl}}
    Вывод программы:
    {\small \VerbatimInput{logs/log_17.log}}


    \section*{Задание 18a}

    Составить программу для определения, является ли первый список подсписком второго.


    Исходный код:
    {\small \VerbatimInput{code/task_18a.pl}}
    Вывод программы:
    {\small \VerbatimInput{logs/log_18a.log}}

    \section*{Задание 18b}
    Второй вариант предыдущей задачи: определить подсписок через предикат объединения списков append.

    Исходный код:
    {\small \VerbatimInput{code/task_18b.pl}}
    Вывод программы:
    {\small \VerbatimInput{logs/log_18b.log}}

    \section*{Задание 19}

    Написать программу, позволяющую найти список, являющийся пересечением двух списков.

    Исходный код:
    {\small \VerbatimInput{code/task_19.pl}}
    Вывод программы:
    {\small \VerbatimInput{logs/log_19.log}}


    \section*{Задание 20}

    Написать программу, позволяющую определить длину строки, введённой пользователем, не используя стандартного предиката str\_len.


    Исходный код:
    {\small \VerbatimInput{code/task_20.pl}}
    Вывод программы:
    {\small \VerbatimInput{logs/log_20.log}}


    \section*{Задание 21}

    Подсчитать, сколько символов находится в файле, определенном пользователем.


    Исходный код:
    {\small \VerbatimInput{code/task_21.pl}}
    Текст файла:
    {\small \VerbatimInput{code/task_21.txt}}
    Вывод программы:
    {\small \VerbatimInput{logs/log_21.log}}


    \section*{Задание 22}

    Подсчитать, сколько слов находится в файле, определенном пользователем.


    Исходный код:
    {\small \VerbatimInput{code/task_22.pl}}
    Вывод программы:
    {\small \VerbatimInput{logs/log_22.log}}


    \section*{Задание 23}

    Файл tree.dat содержит базу данных о плодовых деревьях: яблонях и грушах, включая сорт и тип плодоношения.
    Необходимо запросить у пользователя дерево и тип плодоношения и выдать ему все подходящие сорта. В случае отсутствия подходящего сорта выдать ответ <<Нет данных>>.


    Исходный код:
    {\small \VerbatimInput{code/task_23.pl}}
    Текст файла tree.dat:
    {\small \VerbatimInput{code/tree.dat}}
    Вывод программы:
    {\small \VerbatimInput{logs/log_23.log}}


    \section*{Задание 24}

    В файле tel.dat разместить данные об именах и номерах телефонов, используя предикат телефон(имя, номер).
    Составить программу, позволяющую проводить дополнение базы данных, с помощью примерного диалога:

    Введите имя: Толя

    Введите телефон: 1234567

    Продолжить? да/нет

    По окончании дополнения базы следует сохранить обновлённый файл tel.dat.

    Исходный код:
    {\small \VerbatimInput{code/task_24.pl}}
    Текст файла tel.dat:
    {\small \VerbatimInput{code/tel.dat}}
    Вывод программы:
    {\small \VerbatimInput{logs/log_24.log}}
    Текст файла tel\_updated.dat:
    {\small \VerbatimInput{code/tel_updated.dat}}

\end{document} % конец документа
